\documentclass[11pt]{article}
\usepackage[utf8]{inputenc}

\usepackage{latexsym}
\usepackage{amssymb,amsmath}
\usepackage{graphicx}
\usepackage{sgame}
\usepackage{color}
\usepackage{authblk}

% \usepackage{hyperref}
\usepackage{empheq}
\usepackage{blkarray}
\usepackage{cancel}
\usepackage{enumerate}
\usepackage{times}
\usepackage{array}
\usepackage{lscape}


\usepackage[margin=1in]{geometry}
\newcommand{\newword}[1]{\textbf{\emph{#1}}}

%Arrows
\newcommand{\into}{\hookrightarrow}
\newcommand{\onto}{\twoheadrightarrow}

%Things LaTeX names by appearance, rather than meaning
% By now, I've learned the standard LaTeX names, but I remember they used to give me trouble, so here are some macros
\newcommand{\isom}{\cong} %The isomorphism symbol
\newcommand{\union}{\cup}
\newcommand{\intersection}{\cap}
\newcommand{\bigunion}{\bigcup}
\newcommand{\bigintersection}{\bigcap}
\newcommand{\disjointunion}{\sqcup}
\newcommand{\bigdisjointunion}{\bigsqcup}

\newcommand\numberthis{\addtocounter{equation}{1}\tag{\theequation}}

%Some multiletter functions
\DeclareMathOperator{\Hom}{Hom}
\DeclareMathOperator{\Ext}{Ext}
\DeclareMathOperator{\End}{End}
\DeclareMathOperator{\Tor}{Tor}
\DeclareMathOperator{\Ker}{Ker}
\DeclareMathOperator{\CoKer}{CoKer}
\DeclareMathOperator{\Spec}{Spec}
\DeclareMathOperator{\Proj}{Proj}
\renewcommand{\Im}{\mathop{\mathrm{Im}}}
%Their calligraphic versions; use these for the sheaf constructions
\DeclareMathOperator{\HHom}{\mathcal{H} \textit{om}}
\DeclareMathOperator{\EExt}{\mathcal{E} \textit{xt}}
\DeclareMathOperator{\EEnd}{\mathcal{E} \textit{nd}}
\DeclareMathOperator{\TTor}{\mathcal{T} \textit{or}}
\DeclareMathOperator{\KKer}{\mathcal{K}\textit{er}}
\DeclareMathOperator{\CCoKer}{\mathcal{C} \textit{o}\mathcal{K} \textit{er}}
\newcommand{\IIm}{\mathop{\mathcal{I} \textit{m}}}
\newcommand{\ccH}{\mathscr{H}} %The very curly H

\DeclareMathOperator{\sss}{\mathrm{sunny}}
\DeclareMathOperator{\rrr}{\mathrm{rainy}}
\DeclareMathOperator{\hhh}{\mathrm{hot}}
\DeclareMathOperator{\ccc}{\mathrm{cold}}



%This makes alternating tensors look right in displayed equations
\newcommand{\Alt}{\bigwedge\nolimits}


%Blackboard bold letters.
\renewcommand{\AA}{\mathbb{A}}
\newcommand{\BB}{\mathbb{B}}
\newcommand{\CC}{\mathbb{C}}
\newcommand{\DD}{\mathbb{D}}
\newcommand{\EE}{\mathbb{E}}
\newcommand{\FF}{\mathbb{F}}
\newcommand{\GG}{\mathbb{G}}
\newcommand{\HH}{\mathbb{H}}
\newcommand{\II}{\mathbb{I}}
\newcommand{\JJ}{\mathbb{J}}
\newcommand{\KK}{\mathbb{K}}
\newcommand{\LL}{\mathbb{L}}
\newcommand{\MM}{\mathbb{M}}
\newcommand{\NN}{\mathbb{N}}
\newcommand{\OO}{\mathbb{O}}
\newcommand{\PP}{\mathbb{P}}
\newcommand{\QQ}{\mathbb{Q}}
\newcommand{\RR}{\mathbb{R}}
\renewcommand{\SS}{\mathbb{S}}
\newcommand{\TT}{\mathbb{T}}
\newcommand{\UU}{\mathbb{U}}
\newcommand{\VV}{\mathbb{V}}
\newcommand{\WW}{\mathbb{W}}
\newcommand{\XX}{\mathbb{X}}
\newcommand{\YY}{\mathbb{Y}}
\newcommand{\ZZ}{\mathbb{Z}}

%Calligraphic letters

\newcommand{\cA}{\mathcal{A}}
\newcommand{\cB}{\mathcal{B}}
\newcommand{\cC}{\mathcal{C}}
\newcommand{\cD}{\mathcal{D}}
\newcommand{\cE}{\mathcal{E}}
\newcommand{\cF}{\mathcal{F}}
\newcommand{\cG}{\mathcal{G}}
\newcommand{\cH}{\mathcal{H}}
\newcommand{\cI}{\mathcal{I}}
\newcommand{\cJ}{\mathcal{J}}
\newcommand{\cK}{\mathcal{K}}
\newcommand{\cL}{\mathcal{L}}
\newcommand{\cM}{\mathcal{M}}
\newcommand{\cN}{\mathcal{N}}
\newcommand{\cO}{\mathcal{O}}
\newcommand{\cP}{\mathcal{P}}
\newcommand{\cQ}{\mathcal{Q}}
\newcommand{\cR}{\mathcal{R}}
\newcommand{\cS}{\mathcal{S}}
\newcommand{\cT}{\mathcal{T}}
\newcommand{\cU}{\mathcal{U}}
\newcommand{\cV}{\mathcal{V}}
\newcommand{\cW}{\mathcal{W}}
\newcommand{\cX}{\mathcal{X}}
\newcommand{\cY}{\mathcal{Y}}
\newcommand{\cZ}{\mathcal{Z}}


\DeclareMathOperator{\ord}{ord}
\DeclareMathOperator{\inte}{int}
\DeclareMathOperator{\nhd}{nhd}

\newcommand{\ds}{\displaystyle}
\newcommand{\mc}{\mathcal}
\newcommand{\ol}{\overline}
\newcommand{\modu}{\hspace{-2mm} \mod}

\DeclareMathOperator{\inn}{Inn}
\DeclareMathOperator{\aut}{Aut}
\DeclareMathOperator{\cen}{Center}
\DeclareMathOperator{\im}{Im}
\DeclareMathOperator{\re}{Re}
\DeclareMathOperator{\id}{id}
\DeclareMathOperator{\mor}{Mor}
\DeclareMathOperator{\irr}{Irr}
\DeclareMathOperator{\sgn}{sgn}

\DeclareMathOperator{\cov}{Cov}
\DeclareMathOperator{\var}{Var}

\DeclareMathOperator{\erf}{erf}
%\DeclareMathOperator{\sgn}{sgn}
\DeclareMathOperator{\argmin}{argmin}
\DeclareMathOperator{\argmax}{argmax}

\DeclareMathOperator{\lip}{Lip}

\newcommand{\bbm}{\begin{bmatrix}}
\newcommand{\bpm}{\begin{pmatrix}}
\newcommand{\ebm}{\end{bmatrix}}
\newcommand{\epm}{\end{pmatrix}}

\newcommand{\ddx}[2]{\frac{d #1}{d #2}}
\newcommand{\ddt}[1]{\frac{d #1}{dt}}

 \newcommand{\del}[2]{\frac{\partial #1}{\partial #2}}
 \newcommand{\dsdel}[2]{\displaystyle\frac{\partial #1}{\partial #2}}
 
 \newcommand{\doubledel}[3]{\displaystyle\frac{\partial^2 #1}{\partial #2 \partial #3}}
 \newcommand{\doubledelsame}[2]{\displaystyle\frac{\partial^2 #1}{\partial #2^2}}
  
%newcommand{\ddx}[2]{\frac{d #1}{d #2}}
%\newcommand{\ddt}[1]{\frac{d #1}{dt}}

\newcommand{\dsddx}[2]{\displaystyle\frac{d #1}{d #2}}
\newcommand{\dsddt}[1]{\displaystyle\frac{d #1}{dt}}

\newcommand{\pbderiv}{\ds\del{V}{x_1} \dsddt{x_1} + \ds\del{V}{x_2} \dsddt{x_2}}

\newcommand{\ito}{It\^o \hspace{0.05mm}}
\newcommand{\itos}{It\^os \hspace{0.05mm}}

\newcommand{\gronwall}{Gr\"onwall  \hspace{0.05mm}}
\newcommand{\gronwalls}{Gr\"onwall's  \hspace{0.05mm}}

\newcommand{\tw}{d\tilde{W}_t}
\newcommand{\tws}{d\tilde{W}_s}

\title{Fixed Threshold Mixing Model ({\color{red}Draft})}
\author{\hspace{0pt}\vspace{-30pt}}
\date{Last updated: November 9, 2018}

\bibliographystyle{plain}

%%%%%%%%%%%%%%%%%%%%%%%%%%%
\begin{document}

\maketitle

{\color{red}Chris: I am planning to first add all relevant figures to this doc so that we can then discuss which ones to highlight in a more finalized document -- but let me know if you think of a better way to do this!}

\section{Aim}
The goal of this investigation is to understand how varying one or more or the parameters of the fixed threshold model affects the behavior of single line (A or B) colonies and mixed colonies.
\\

{\color{red}
Here I think we might want to insert a list of specific patterns that (we think) we observe in the data:
\begin{itemize}
    \item Different mean frequencies of task 1 performance between line A and B ants
    \item etc.
\end{itemize}
so that we can agree on what it is that we want the model to do.
}

\section{Plan}
\subsection{Original model in Ulrich et al. \cite{ulrich2018}}
\begin{table}[h!] \small
  \begin{center}
    \begin{tabular}{|c|>{\centering}p{2.05in}|c|c|} 
      \hline
      \textbf{Parameter} & \textbf{Definition} & \textbf{Model values} & \textbf{Notes} \\ \hline
      $n$ & No. of individuals & $1\leq n \leq16$ & \\ \hline
      $m$ & No. of tasks & $m = 2$ & \\ \hline
      $\delta_j$ & Task-specific demand rate & $\delta_j = \delta = 0.6$ & Assumed to be the same across all tasks \\ \hline
      $\alpha_j$ & Task-specific performance efficiency & $\alpha_j = \alpha = m(=2) $ & Assumed to be the same across all tasks \\ \hline
      $\mu_j$ & Task-specific mean threshold & $\mu_j = \mu = 10$ & Assumed to be the same across all tasks \\ \hline
      $\sigma_j$ & Task-specific threshold variation & $0 \leq \sigma_j = \sigma \leq 0.5$ & Assumed to be the same across all tasks \\ \hline
      $\eta$ & Threshold stochasticity & $1 \leq \eta \leq 30 $ & \\ \hline
      $\tau$ & Quit probability & $\tau = 0.2$ & \\ \hline
    \end{tabular}
    \caption{Parameterization of the fixed threshold model in Ulrich et al. \cite{ulrich2018}. Note that the model values only include those in Fig.~3 of \cite{ulrich2018}.}
    \label{tab:table1}
  \end{center}
\end{table}

\newpage
\subsection{\textit{Varying} across lines and \textit{fixing} across tasks}
\begin{table}[hbt!] \small
  \begin{center}
    \begin{tabular}{|c|>{\centering}m{0.65in}|>{\centering}m{1.15in}|m{3.5in}|} 
      \hline
      \textbf{Parameter} & \textbf{Varied?} & \textbf{Model Value(s)} & \textbf{Notes / Biological Interpretation} \\ \hline
      $n$ & Fixed & $n = 4, 16$ & Based on colony sizes used in the mixing experiments \\ \hline
      $m$ & Fixed & $m = 2$ & For simplicity; no compelling reason to change from \cite{ulrich2018} \\ \hline
      $\delta$ & Fixed & $\delta = 0.6$ & Stimulus increase rate should not depend on line\\ \hline
      $\alpha$ & By line  & $\alpha^A = 2,\alpha^B = 6$ & Line B ants are be more efficient than line A ants at both tasks \\ \hline
      $\mu$ & By line  & $\mu^A = 10,\mu^B = 20 $ & Line B ants have higher mean thresholds than line A ants for both tasks; creates a bimodal distribution of thresholds in the mixed case \\ \hline
      $\sigma$ & By line & $\sigma^A = 0.1, \sigma^B = 0.3$ & Line B ants have a greater range of internal thresholds than line A ants; seems unlikely since the variance in RMSD does not appear to differ between lines A and B, but included here for thoroughness \\ \hline
      $\eta$ & By line & $\eta^A = 7, \eta^B = 14 $ &  Line B ants respond more deterministically to both stimuli than line A ants\\ \hline
      $\tau$ & By line & $\tau^A = 0.2,\tau^B = 0.6 $ & Line B ants tend to spend less time on a given task than line A ants; could be correlated to their relative cycle lengths (A has slower cycles than B) \\ \hline
    \end{tabular}
    \caption{Varying parameters by line.}
    \label{tab:table2}
  \end{center}
\end{table}

\subsection{\textit{Fixing} line-specific parameters and \textit{varying} task-specific parameters}

\begin{table}[hbt!] \small
  \begin{center}
    \begin{tabular}{|c|>{\centering}m{0.65in}|>{\centering}m{1.15in}|m{3.5in}|} 
      \hline
      \textbf{Parameter} & \textbf{Varied?} & \textbf{Model Value(s)} & \textbf{Notes / Biological Interpretation} \\ \hline
      $n$ & Fixed & $n = 4, 16$ & Based on colony sizes used in the mixing experiments \\ \hline
      $m$ & Fixed & $m = 2$ & For simplicity; no compelling reason to change from \cite{ulrich2018} \\ \hline
      $\delta$ & By task & $\delta_1 = 0.6, \delta_2 = 1.8 $ & Demand for task 2 increases more rapidly than that for task 1; $\delta$ could capture the difference between A larvae vs. B larvae? \\ \hline
      $\alpha$ & By task  & $\alpha_1 = 2,\alpha_2 = 6$ & Ants are more efficient at task 2 than at task 1 \\ \hline
      $\mu$ & By task  & $\mu_1 = 10,\mu_2 = 20 $ & The mean threshold for task 2 is higher than that for task 1\\ \hline
      $\sigma$ & By task & $\sigma_1 = 0.1, \sigma_2 = 0.3$ & Ants vary more in their international thresholds for task 2 than those for task 1 \\ \hline
      $\eta$ & Fixed & $\eta = 7$ & Stochasticity in behavior is inherent to the ants, not tasks\\ \hline
      $\tau$ & By task & $\tau_1 = 0.2,\tau_2 = 0.6 $ & Ants have a shorter period for task 2 relative to task 1 (e.g., task 2 is more taxing so they need to take more breaks) \\ \hline
    \end{tabular}
    \caption{Varying parameters by task.}
    \label{tab:table3}
  \end{center}
\end{table}

\begin{thebibliography}{99}

\bibitem{ulrich2018} Y. Ulrich, J. Saragosti, C. K. Tokita, C. E. Tarnita, D. J. C. Kronauer, ``Fitness benefits and emergent division of labour at the onset of group living,'' \textit{Nature}, vol. 560, pp. 635-638, Aug. 2018.

\end{thebibliography}

\end{document}
%%%%%%%%%%%%%%%%%%%%%%%%%%%

